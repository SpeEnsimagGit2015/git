\documentclass[a4paper, 12pt]{article}


\usepackage[french]{babel}
\usepackage[utf8]{inputenc}
\usepackage[T1]{fontenc}
\usepackage{lmodern}
\usepackage{listings}
\usepackage{graphicx}
\usepackage{amsmath}
\usepackage{amsfonts}
\usepackage{amssymb}
\usepackage{caption}
\usepackage{subcaption}
\usepackage[usenames,dvipsnames]{xcolor}


\setcounter{secnumdepth}{4}
% TAILLE DES PAGES (A4 serré)

\setlength{\parindent}{0pt}
\setlength{\parskip}{1ex}
\setlength{\textwidth}{17cm}
\setlength{\textheight}{24cm}
\setlength{\oddsidemargin}{-.7cm}
\setlength{\evensidemargin}{-.7cm}
\setlength{\topmargin}{-.5in}

% Commandes de mise en page
\newcommand{\fichier}[1]{\emph{#1}}
\newcommand{\nom}[1]{\emph{#1}}
\newcommand{\Fig}[1]{Fig \ref{#1} p. \pageref{#1}}
\newcommand{\itemi}{\item[$\bullet$]}

% Commandes de maths
\newcommand{\fonction}[3]{#1 : #2 \to #3}
\newcommand{\intr}[2]{\left[ #1 ; #2 \right]}
\newcommand{\intn}[2]{\left[\![ #1 ; #2 \right]\!]}
\newcommand{\intro}[2]{\left] #1 ; #2 \right[}
\newcommand{\intrsod}[2]{\left[ #1 ; #2 \right[}
\newcommand{\ps}[2]{\langle #1, #2 \rangle}
\newcommand{\mdelta}[1]{\boldsymbol{\delta_{#1}}}
%% \newcommand{\mdelta}[1]{\delta_{\textbf{#1}}}

\pagenumbering{arabic}
\graphicspath{{images/}}

\title{Bilan} 
\author{Antoine Delaite \and Louis--Alexandre Stuber \and Guillaume Pages \and Rémi Galan-Alphonso \and Rémi Lespinet}
\date{}

\begin{document}

\maketitle

\section{Introdution}

Le projet GL a été un challenge intéressant. Les trois semaines
ont été intenses, et le groupe a connu une évolution dans
l'organisation et la communication durant cette période.  C'est
pouquoi nous avons choisi de présenter le déroulement de notre projet
sous forme chronologique, en découpant selon nos différents sprints.
Pour chaque sprint, on détaille son contenu et la façon dont nous nous
sommes organisés, les ressentis et une approche critique sur nos choix
en expliquant ce que l'on aurait pu améliorer.


\section{Organisation}

\subsection{Présentation de l'équipe}

Dans cette partie, nous commençons par présenter l'équipe et les contraintes
induits par chacun des membres pour justifier les choix qui ont été pris
tout au long du projet.

L'une des contrainte majeure est que Guillaume avait cours normalement
pendant toute la durée du projet, en particulier il n'a pu être
présent à tous les suivis ni au cours d'explication.  De plus il avait
2 autres projets à faire pendant la durée du projet GL.

Vicente vient d'une école au Brésil où il n'a pas fait de Théorie des Langages,
ce qui limite la compréhension du sujet, de plus il ne connaissait pas
les outils avec lesquels nous avons l'habitude de travailler.

\subsection{Organisation selon les contraintes}

Dès le premier jour, tout le monde était d'accord pour que Vicente s'occupe
de la parie mathématique. C'était un choix juste car c'est la partie
avec laquelle il se sentait le plus à l'aise, et qui ne nécessitait pas
de connaissances particulières en Théorie des Langages ni un grand usage
du logiciel git, puisque cette partie est très indépendante du reste
et n'a pas besoin d'être partagée avec tout le monde.

Organisations suivant les contraintes : Comme Guillaume ne pouvait pas
être là à plein temps, nous avons décidé tous ensemble qu’il tiendrait le rôle
de support et qu’il ne sera pas sur une partie à part entière pour
éviter de prendre du retard si jamais il ne peut pas avancer. C’était
un très bon choix car Guillaume a une très grande capacité de
compréhension et pouvait vraiment bien comprendre toutes les parties
du projet. Il a ainsi pu aider tout le reste de l’équipe durant tout
le projet.

%Pour autant, nous ne sommes pas satisfaits pleinement,

\subsection{Les Sprints}

\subsubsection{HelloWorld}

\paragraph*{Le déroulement :}
Dans la première semaine, nous voulions démarrer au plus vite le projet,
nous avons décidé d'une organisation à très court terme, rapidement.
Cela nous a permis d'avancer assez vite dans un premier temps, mais
nous avons été rattrapés par ce manque de vision a long terme quand il
a fallu passer à une autre partie. Il n'y a pas assez de communication
au sein de l'équipe au sens où il n'y avait pas assez de réunions et on
travaillait au jour le jour : on ne savait pas quelle tâche on allait
faire après.

Nous avons décidé de faire toute la partie A en entière y compris les
objets, \nom{casts} et \nom{instanceof}, car nous avons jugé que la
tâche était assez répétitive et ne prenait pas longtemps à faire en
une fois, par rapport au temps que cela aurait pris en y revenant
et en se remettant dans le bain à chaque fois.

Après coup, nous pensons que c'était un bon choix, car nous n'avons
par la suite quasiment pas eu besoin de revenir dessus (à part quelques
petites erreurs). De plus cette partie a été réalisée en pair programming
ce qui a grandement limité les erreurs.

Nous n'avions en revanche pas commencé la partie B ce qui représente un
léger retard sur le planning.

\paragraph*{Le suivi :}
Lors du suivi on a réalisé que notre organisation était très insuffisante.
A l'aide des deux professeurs d'agilité, nous avons pu nous organiser
et adapter la méthode agile aux contraintes de notre groupe.
Par exemple, il était impossible de faire une mêlée matinale tous les 5
car Guillaume avait cours la majorité des matinées. Nous organisions
donc une réunion le soir, la majorité du temps lors d'une audioconférence.

\paragraph*{Réaction au suivi :}

Nous avons directement dédié une matinée à l'élaboration d'un paperboard
en spécifiant les userstories. Même si les userstories étaient un peu
décalées de la réalité, cela a permis d'avoir une meilleure vision
du projet, et d'avoir un premier jet de la répartition des tâches


\subsubsection{Sans objets (rendu intermédiaire)}


\paragraph*{Le déroulement :}
Dans cette nouvelle organisation, Vicente a continué d'implémenter
les fonctions de la partie Maths. Nous avons divisé le reste de l'équipe
en deux : Guillaume et Antoine s'occupaient de la partie B ainsi que des tests
pour cette partie, Rémi et Xin, quant-à eux, s'occupaient de la partie C.
Dans chacune des deux équipe, il y avait une personne qui s'occupait plutôt
de l'implémentation et l'autre du test. Il était préférable que Guillaume
soit sur les tests, au vu de ses disponibilités. Le reste des tâches
fut décidé selon les affinités.

L'organisation était bien meilleure, nous avions a la fois une bonne vision
globale du projet, notamment grâce au planning des tâches, et les réunions
nous permettaient de nous tenir au courant de l'avancement de chacun.
Cette meilleure communication générale nous permettait d'échanger plus
facilement lorsque nous rencontrions des erreurs.

Par exemple, les erreurs trouvées sur la partie A dans la partie B
remontaient très vite et étaient corrigées très rapidement. De même
pour les erreurs trouvées sur la partie B grâce à la partie C. Il y
avait une excellente entente et disponibilité de chacun, Et une
volonté d'aider, afin que personne ne soit ralenti dans son travail.

Dans l'ensemble, nous étions assez satisfait du rendu pour la partie B
et les tests.  Pour la Partie C en revanche, la structure de test
n'était pas encore bien mise en place, et la majorité des fichiers de
tests écrits n'a pas pu être testé en pratique.

Avec une meilleure organisation en début de semaine, nous aurions pu
anticiper le problème et faire de l'automatisation des tests une
priorité.

\paragraph*{Le suivi :}
Les mauvais résultats de la partie C n'avaient pas été anticipés,
ceux-ci ont confirmé les lacunes au niveau des tests sur cette partie.

\paragraph*{Réaction au suivi :}
Pour remédier à la situation, des scripts de tests ont aussitôt été
créés.

\subsubsection{Avec objets (rendu final)}

\paragraph*{Le déroulement :}
Dès le départ, nous avons fait une réunion pour actualiser nos
attentes. Nous avons donc décidé, afin d'avoir une marge pour les
tests, de ne pas faire les \nom{casts} et \nom{instanceof}.

Lors de cette décision, il a aussi été décidé que Guillaume rejoigne
Vicente sur la partie Maths, car il y rencontrait des problèmes, et
que les prévisions faites allaient normalement être respectées.

Le mercredi, Rémi est tombé malade, assez durement, et a du rester au
lit pendant pratiquement 3 jours. Il s'agissait d'un risque que l'on
avait envisagé, mais que l'on avait négligé. Le problème fut que
l'implémentation de la Partie C n'a pas vraiment avancé pendant ce
temps.  Au vu de la disposition que l'on avait adopté, il était
difficile pour les autres membres de l'équipe de trouver du temps pour
se plonger dans cette partie parallèlement aux autres
activités. Malheureusement, il s'agissait d'une tâche critique, sur
laquelle nous avions très peu de marge. Nous avons donc décidé de
sacrifier une partie de la documentation de la partie C.

Cela a tout-de-même mis en évidence qu'au moins une partie du projet
reposait trop sur un seul membre. Nous aurions pu palier à ce
problème en utilisant le pair programming, si il y avait eu moins
de contrainte au niveau de l'équipe.

\section{Bilan}

\subsection{Réactions le jour du rendu}

Nous avons pu arrêter de coder et rendre quelque chose de propre le
dernier jour, on était effectivement dans les temps par rapport aux
dernières prévisions.

\subsection{Conclusion}

Globalement, nous sommes satisfaits, le projet s'est très bien passé.
Nous avons rencontré des difficultés que nous avons essayé de gérer
du mieux possible. L'ambiance au sein du groupe a toujours été
au beau fixe. La motivation et l'enthousiasme étaient au rendez
vous. On note seulement deux moments difficiles : en début de projet et
lorsque Rémi était malade. Le stress et la fatigue n'étaient pas
particulièrement présents. Nous n'avons pas de regrets particulier.
Compte tenu des circonstances, nous pensons que nous n'aurions pas pu
obtenir un meilleur résultat.


\section{Enseignements}



\begin{itemize}
\item Organisation : Le projet nous a appris à organiser
un groupe de 5, c'était pour nous le premier projet de
cette envergure.
\item Méthode agile : Le projet nous a initié à la méthode
agile qui est une nouvelle méthodologie de travail pour nous
, mêlées, user stories, scrum master...
\item S'adapter : Le projet nous a appris à nous adapter
à la diversité des personnalités et des méthodes de
travail de chacun.
\item Faire face a l'imprévu : Le projet nous a appris à prendre
des décisions dans l'urgence.
\item Etre détendu : le meilleur moyen de surmonter les quelques
problèmes qu'on a eu était de ne pas stresser.
\item Impliqué : Il était important de se donner a fond dans le
projet et d'être concentré.
\item Autonome : La quantité de travail nous pousse à prendre
des responsabilités et prendre des choix sur notre partie.
\item Communication : Le projet nous a appris à présenter ce
que l'on a fait et avoir une communication presque professionnelle
à l'intérieur d'un groupe.
\item Role des tests : Dans ce projet, il faut être très rigoureux
au niveau des tests, il s'agit d'une étape clé avant de rendre un
produit final.
\item Découverte de nouveaux outils : Nous avons découvert l'outil
maven pour la compilation, cobertura, JUnit pour les tests, antlr
pour la génération de grammaire, et netbeans comme IDE.
\item Théorie des langages : Nous avons parfait nos compétentes
en assembleur, et nous avons une meilleure idée du fonctionnement
d'un compilateur, qui est un outil que nous utilisons au quotidien.
\end{itemize}


\end{document}

% TODO : parler du scrum master
